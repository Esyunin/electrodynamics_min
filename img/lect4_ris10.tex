\documentclass[tikz]{standalone}
\usetikzlibrary{arrows}
\usetikzlibrary{3d}
\usepackage{cmap}
\usepackage[T2A]{fontenc}
\usepackage[utf8x]{inputenc}
\usepackage[english, russian]{babel}
\usepackage{tikz,amssymb,amsmath}
\usetikzlibrary
    {
        decorations.pathmorphing,
        decorations.pathreplacing,
        decorations.markings,
        shapes.misc,
        patterns,
        calc,
        scopes,
        arrows,
        fadings,
        through,
        shapes.misc,
        arrows.meta,
        3d,
        quotes,
        angles,
        babel
    }


\begin{document}
\begin{tikzpicture}[scale=2,
    %Option for nice arrows
    >=stealth, %
    inner sep=0pt, outer sep=2pt,%
    axis/.style={thick,->},
    wave/.style={thick,color=#1,smooth},
    polaroid/.style={fill=black!60!white, opacity=0.3},
    interface1/.style={draw=gray!60,
        % The border decoration is a path replacing decorator. 
        % For the interface style we want to draw the original path.
        % The postaction option is therefore used to ensure that the
        % border decoration is drawn *after* the original path.
        postaction={draw=gray!60,decorate,decoration={border,angle=-135,
                    amplitude=0.2cm,segment length=0.5mm}}}, 
    interface/.style={
        pattern = north east lines,
        draw    = none,
        pattern color=blue!60,          
    },
]

\newcommand{\volnovod}[5]{ %#1 #2 (lx,ly) -- #3 #4 (rx,ry) 
    \draw[interface] (#1,#2) rectangle (#3,#2-#5);

    \draw[interface] (#1,#4) rectangle (#3,#4+#5);
    \draw[black] (#1,#2) -- (#3,#2);
    \draw[black] (#1,#4) -- (#3,#4);
}
% Draw line annotation
% Input:
%   #1 Line offset (optional)
%   #2 Line angle
%   #3 Line length
%   #5 Line label
% Example:
%   \lineann[1]{30}{2}{$L_1$}
\newcommand{\lineann}[4][0.5]{%
    \begin{scope}[rotate=#2, blue,inner sep=2pt]
        \draw[dashed, blue!40] (0,0) -- +(0,#1)
            node [coordinate, near end] (a) {};
        \draw[dashed, blue!40] (#3,0) -- +(0,#1)
            node [coordinate, near end] (b) {};
        \draw[|<->|] (a) -- node[fill=white] {#4} (b);
    \end{scope}
}

    % Colors
    \colorlet{darkgreen}{green!50!black}
    \colorlet{lightgreen}{green!80!black}
    \colorlet{darkred}{red!50!black}
    \colorlet{lightred}{red!80!black}
    % \draw[thick] (0,0) rectangle (1.5,1);
    % \draw (1.5,0)  node[below] {$a$};
    % \draw (0,1)  node[left] {$b$};
    % \draw (0,0)  node[anchor=north east,yshift=-0.5em, xshift=-.5em] {$0$};

    \draw[white] (-0.25,-0.25);
    % \draw (0.75,0)  node[below, yshift=-.25cm] {$\varkappa(a,b)$};
    
    \draw[->] (0,0) -- (4.5,0) node [right] {$z$};
    \draw[->] (0,0) -- (0,1.5) node [right] {$y$};

    \volnovod{0}{0}{4.25}{1}{0.1};

    \draw[fill=white,draw=none] (0,0) circle (2.5pt) node {$\bigotimes$} node[left,xshift=-.5em,yshift=-.5em] {$x$};
    % \draw (3,0.5) circle (.5cm);
    % \draw[->] (3,0.5) -- node[sloped,above] {r} ++(30:0.5);
    % \draw (3,0)  node[below, yshift=-.25cm] {$\varkappa(r)$};

    \foreach \x in {0.25,0.5,0.6,0.7,0.75,0.8,0.9,1,1.25}{
        \draw[postaction=decorate,decoration={
        markings,
        mark=at position 0.525 with {\arrow{>}}}] (\x,0) -- ++ (0,1);
    }

    \foreach \x in {0.25,0.5,0.6,0.7,0.75,0.8,0.9,1,1.25}{
        \draw[postaction=decorate,decoration={
        markings,
        mark=at position 0.525 with {\arrow{<}}},xshift=1.5cm] (\x,0) -- ++ (0,1);
    }




    \foreach \x in {0.25,0.5,0.6,0.7,0.75,0.8,0.9,1,1.25}{
        \draw[postaction=decorate,decoration={
        markings,
        mark=at position 0.525 with {\arrow{>}}},xshift=3cm] (\x,0) -- ++ (0,1);
    }


    \foreach \xxx in {0.75,0.75+3}{
        \foreach \yyy in {0.2,0.4,0.6,0.8}{
            \draw[fill=white,draw=none] (\xxx,\yyy) circle (1.5pt) node[,scale=0.7] {$\bigodot$};
            % \breakforeach
        }
        \foreach \yyy in {0.25,0.5,0.75}{
            \draw[fill=white,draw=none] (\xxx-0.2,\yyy) circle (1.5pt) node[,scale=0.7] {$\bigodot$};            
            \draw[fill=white,draw=none] (\xxx+0.2,\yyy) circle (1.5pt) node[,scale=0.7] {$\bigodot$};
            % \breakforeach
        }
        \foreach \yyy in {0.33,0.66}{
            \draw[fill=white,draw=none] (\xxx-0.4,\yyy) circle (1.5pt) node[,scale=0.7] {$\bigodot$};            
            \draw[fill=white,draw=none] (\xxx+0.4,\yyy) circle (1.5pt) node[,scale=0.7] {$\bigodot$};
            % \breakforeach
        }
    }

    \foreach \xxx in {0.75+1.5}{
        \foreach \yyy in {0.2,0.4,0.6,0.8}{
            \draw[fill=white,draw=none] (\xxx,\yyy) circle (1.5pt) node[,scale=0.7] {$\bigotimes$};
            % \breakforeach
        }
        \foreach \yyy in {0.25,0.5,0.75}{
            \draw[fill=white,draw=none] (\xxx-0.2,\yyy) circle (1.5pt) node[,scale=0.7] {$\bigotimes$};            
            \draw[fill=white,draw=none] (\xxx+0.2,\yyy) circle (1.5pt) node[,scale=0.7] {$\bigotimes$};
            % \breakforeach
        }
        \foreach \yyy in {0.33,0.66}{
            \draw[fill=white,draw=none] (\xxx-0.4,\yyy) circle (1.5pt) node[,scale=0.7] {$\bigotimes$};            
            \draw[fill=white,draw=none] (\xxx+0.4,\yyy) circle (1.5pt) node[,scale=0.7] {$\bigotimes$};
            % \breakforeach
        }
    }
    % \foreach \y in {0.25,0.5,0.75}{
    %     \draw[dashed] (0.1,\y) -- ++(1.3,0);
    %     % \draw[postaction=decorate,decoration={
    %     % markings,
    %     % mark=at position 0.5 with {\arrow{>}}}] (\x,0) -- ++ (0,1);
    % }

    \draw (3.2,1.5) node {$\bigodot$,$\bigotimes$ -- поле $\vec{H}$}; 
    \draw (1.5,1.5) node {$\uparrow,\downarrow$ -- поле $\vec{E}$}; 

    \begin{scope}[xshift=0.75cm]
        \lineann[-0.5]{0}{3}{$\lambda_\text{в}$};
    \end{scope}
\end{tikzpicture}
\end{document}