\documentclass[a4paper,14pt]{extarticle}

\usepackage{cmap}
\usepackage[T2A]{fontenc}
\usepackage[utf8x]{inputenc}
\usepackage[english, russian]{babel}

\usepackage{misccorr} % в заголовках появляется точка, но при ссылке на них ее нет
\usepackage{amssymb,amsfonts,amsmath,amsthm}  
\usepackage{indentfirst}
\usepackage[usenames,dvipsnames]{color} 
\usepackage[unicode,hidelinks]{hyperref}
% \hypersetup{%
%     pdfborder = {0 0 0}
% }
\usepackage{physics}

\usepackage{makecell,multirow} 
\usepackage{ulem}
\usepackage{graphicx,wrapfig}
\graphicspath{{img/}}
\usepackage{geometry}
\geometry{left=2cm,right=2cm,top=2cm,bottom=2cm,bindingoffset=0cm}
\usepackage{fancyhdr} 
\renewcommand{\phi}{\varphi}
\renewcommand{\epsilon}{\varepsilon}
\renewcommand{\kappa}{\varkappa}
\linespread{1.05} 
\frenchspacing 
\begin{document}
	\begin{center}
		\Large \textbf{Программа минимум по электродинамике}
	\end{center}
		\textit{Записать формулы и построить графики (без вывода), объяснить используемые в них обозначения: дать требуемые определения}
	\begin{enumerate}
		\item 
		\hyperlink{num1}{Запись функции, определяющей зависимость полей и векторных потенциалов гармонической плоской волны в линии передачи от времени $t$ и продольной координаты $z$. Понятия частоты, временного периода, продольного волнового числа, длины волны, фазовой и групповой скорости.}
%		\hyperref[num1]{Запись функции, определяющей зависимость полей и векторных потенциалов гармонической плоской волны в линии передачи от времени $t$ и продольной координаты $z$. Понятия частоты, временного периода, продольного волнового числа, длины волны, фазовой и групповой скорости.}
		
		\item 
		\hyperlink{num2}{Волновое уравнение для векторного потенциала в отсутствие источников при произвольной и гармонической зависимости от времени. Дифференциальное уравнение для функций поперечных координат $\phi^{(e)}$ и $\phi^{(m)}$. Понятие поперечного волнового числа.}
		
		\item 
		\hyperlink{num3}{Понятие о TE, ТМ и ТЕМ волнах. Импедансная связь поперечных компонент полей. Определение поперечного волнового импеданса.}
		\item
		\hyperlink{num4}{Граничные условия для полей и функций  $\phi^{(e)}$ и $\phi^{(m)}$ в линиях передачи с идеально	
		проводящими границами. Математическая формулировка задачи отыскания собственных волн различных типов в идеальной линии.}
		\item 
		Дисперсионное уравнение для волн в идеальных линиях. Понятие критической частоты и критической длины волны. Графики зависимости полей от продольной координаты в различные моменты времени при частотах, больших или меньших критической. Зависимости длины волны, фазовой и групповой скорости в линии передачи от частоты.
		\item
		В каких линиях могут существовать главные (TEM) волны? Поля TEM волны в	коаксиальной линии.
		\item 
		Спектр поперечных волновых чисел прямоугольного волновода. Низшая мода (поперечное волновое число, графики поля, картина силовых линий). Низшая мода круглого волновода (поперечное волновое число, картина силовых линий).
		\item
		Причины затухания волн в линиях передачи. Описание затухания, обусловленного потерями энергии в заполняющей среде. Графики зависимости поля в линии передачи с потерями от продольной координаты в различные моменты времени.
		\item 
		Описание главных волн в линиях передачи в терминах тока и напряжения: определения величин тока и напряжения, погонной емкости и индуктивности, \underline{определения} волнового сопротивления, импеданса нагрузки, импеданса в любом сечении линии с произвольной нагрузкой на конце.
		\item 
		Коэффициент отражения волны от нагрузки на конце линии. Понятие согласования линии с нагрузкой.
		\item 
		Спектр собственных частот идеального прямоугольного резонатора. Низшая мода прямоугольного резонатора (собственная частота, структура поля).
		\item 
		Причины затухания колебаний в реальных резонаторах. Описание затухания, обусловленного потерями энергии в заполняющей среде. График зависимости поля собственного колебания в реальном резонаторе от времени
		\item 
		Представление полей, создаваемых в волноводе заданными сторонними токами, в виде суперпозиции полей собственных мод (общий вид формул возбуждения волноводов)
		\item 
		Представление полей, создаваемых в резонаторе заданными сторонними токами, в виде суперпозиции полей собственных колебаний (общий вид формул возбуждения резонатора). Резонансные свойства полей.
		\item 
		Способы возбуждения волноводов и резонаторов при помощи штыря и петли.
		\item 
		Определения дифференциального и полного сечений рассеяния тела. Выражение для амплитуды поля и плотности потока энергии рассеянной волны в дальней зоне через дифференциальное сечение рассеяния.
		\item 
		Условие применимости приближения геометрической оптики в задачах дифракции.
		
	\end{enumerate}
	
	Задачи № 10.1 (а), 10.2, 10.4(а,б), 10.5(а,б), 10.7, 10.8, 10.16, 10.18(6), 10.19, 10.22, 10.23, 10.31(а,б), 10.33 (резонатор без плазмы), 10.35(a), 10.36, 10.38, 10.48, 11.1(1,2,3)
	(В.Б. Гильденбург, М.А. Миллер, Сборник задач по электродинамике, 2001 г.).
	\newpage
	\hypertarget{num1}{}
	Для плоской гармонической волны (TEM) функция, определяющая зависимость полей, задается следующим образом 
	
	$$\vec{A}^e = \phi^e(\vec{r}_\perp)e^{-ihz}\vec{z}_0$$
	Где $\vec{A}^e$ -- векторный потенциал поля, а $\phi^e(\vec{r}_\perp)$ называется поперечной волновой функцией. Поля $\vec{E}$ и $\vec{H}$ определяются следующим образом
	$$\vec{H}=\frac{1}{\mu} rot\vec{A}^e $$
	$$\vec{E}=-\frac{1}{c} \pdv{\vec{A}^e}{t}-\nabla{\phi} $$
	Где $\phi$ -- скалярный потенциал поля. 
	Используя условие калибровки Лоренца
	$$div\vec{A}^e+\frac{\epsilon\mu}{c}i\omega\phi=0$$
	Получим выражения для нахождения полей $\vec{E}$ и $\vec{H}$ в случае гармонической волны:
	$$\vec{H}=\frac{1}{\mu} rot\vec{A}^e $$
	$$\vec{E}=\frac{1}{i k_0\epsilon\mu}(\nabla div + k^2)\vec{A}^e $$
	$$k_0=\frac{\omega}{c}, \quad k=\frac{\omega}{c}\sqrt{\epsilon\mu}$$
	
	Понятие частоты $\omega$ -- равна количеству повторений или возникновения событий (процессов) в единицу времени.
	
	Понятие временного периода $T = \frac{2\pi}{\omega}$ -- время, за которое совершается полное колебание.
	
	Понятие продольного волнового числа $h=\frac{2\pi}{\lambda}$ -- волновым числом  называется быстрота роста фазы волны по пространственной продольной координате.
	
	Понятие длины волны $\lambda$ -- пространственный период колебаний. Расстояние между двумя ближайшими друг к другу точками в пространстве, в которых колебания происходят в одинаковой фазе.
	
	Понятие фазовой $v_{\text{ф}}$ и групповой скорости $v_{\text{гр}}$. Фазовая скорость -- скорость перемещения поверхности постоянной фазы. Групповая скорость -- скорость перемещения квазимонохроматического пакета.
	$$v_{\text{ф}}=\frac{\omega}{h}, \quad v_{\text{гр}}=\pdv{\omega}{h} \Bigl|_{\omega=\omega_0}$$
	Где $\omega_0$ -- несущая частота группового пакета.
	
	\newpage
	\hypertarget{num2}{}
	Волновое уравнение для векторного потенциала в случае произвольной зависимости от времени и отсутствия сторонних источников
	$$\Delta\vec{A} -\frac{\epsilon\mu}{c^2}\pdv[2]{\vec{A}}{t}=0$$ 
	Волновое уравнение для векторного потенциала в случае гармонической зависимости от времени и отсутствия сторонних источников
	$$\Delta\vec{A} + k^2 \vec{A}=0$$
	$$\pdv{}{t} \to i\omega, \quad k^2=\frac{\omega^2}{c^2}\epsilon\mu$$
	Дифференциальное уравнение для функций поперечных координат $\phi^{(e)}$ и $\phi^{(m)}$. Понятие поперечного волнового числа.
	$$\vec{A}^e = \phi^e(\vec{r}_\perp)e^{-ihz}\vec{z}_0$$
	$$\Delta = \pdv[2]{}{x}+\pdv[2]{}{y}+\pdv[2]{}{z}=\Delta_\perp+\pdv[2]{}{z}$$ -- для декартовой системы координат.
	$$\Delta A_z^e + k^2 A_z^e=\Delta_\perp\phi^e + (k^2-h^2)\phi^e=0$$
	Тогда дифференциальное уравнение для функций поперечных координат $\phi^{(e)}$ и $\phi^{(m)}$  выглядит следующим образом
	$$\Delta_\perp\phi^{(e,m)} + \kappa^2\phi^{(e,m)}=0$$
	Где $\kappa$ -- поперечное волновое число. 	
	Если функции $\phi^{(e)}$ и $\phi^{(m)}$ удовлетворяют двумерному уравнению Гельмгольца, то поля удовлетворяют уравнению Максвелла.
	
	\newpage
	\hypertarget{num3}{}
	Используем выражения для полей через векторный потенциал
	$$\vec{H}=\frac{1}{\mu} rot\vec{A}^e $$
	$$\vec{E}=\frac{1}{k_0\epsilon\mu}(\nabla div + k^2)\vec{A}^e $$
	Вычислим $\nabla div \vec{A}^e$ и $rot \vec{A}^e$ при условии $\vec{A}^e = \phi^e(\vec{r}_\perp)e^{-ihz}\vec{z}_0$
	$$div \vec{A}^e = -ih\phi^e(\vec{r}_\perp)e^{-ihz}$$
	$$\nabla div \vec{A}^e = (-h^2\phi^e\vec{z}_0-ih\nabla_\perp\phi^e)e^{-ihz}$$
	$$rot \vec{A}^e = [\nabla A^e_z,\vec{z}_0] = [\nabla_\perp\phi^e,\vec{z}_0]e^{-ihz}$$
	Тогда получим следующие выражения для комплексных амплитуд полей TM волны 
	\begin{displaymath}
	e^{-ihz}\cdot
	\begin{cases}
	$$\displaystyle E_z = \frac{\kappa^2}{k_0\epsilon\mu}\phi^e(\vec{r}_\perp)$$ \\
	$$\displaystyle\vec{E}_\perp = -\frac{h}{k_0\epsilon\mu}\nabla_\perp\phi^e(\vec{r}_\perp)$$ \\
	$$\displaystyle\vec{H}_\perp = \frac{1}{\mu}[\nabla_\perp\phi^e(\vec{r}_\perp,\vec{z}_0)]$$ \\
	$$\displaystyle H_z = 0$$
	\end{cases}
	\end{displaymath}

	TM - поперечная магнитная волна. Магнитное поле чисто поперечно пути распространения, но поле $\vec{E}=\vec{E}_\parallel+\vec{E}_\perp$ имеет продольную и поперечную составляющую.
	
	Уравнения Максвелла симметричны относительно полей, но мы получили неравноправные выражения для векторов. Это объясняется тем, что мы нашли одно из решений. Воспользовавшись принципом двойственности $\vec{E}\to\vec{H}$, $\vec{H}\to -\vec{E}$, можно получить выражения  для комплексных амплитуд полей TE волны. 
	\begin{displaymath}
	e^{-ihz}\cdot
	\begin{cases}
	$$\displaystyle H_z = \frac{\kappa^2}{k_0\epsilon\mu}\phi^e(\vec{r}_\perp)$$ \\
	$$\displaystyle\vec{H}_\perp = -\frac{h}{k_0\epsilon\mu}\nabla_\perp\phi^e(\vec{r}_\perp)$$ \\
	$$\displaystyle\vec{E}_\perp = -\frac{1}{\epsilon}[\nabla_\perp\phi^e(\vec{r}_\perp,\vec{z}_0)]$$ \\
	$$\displaystyle E_z = 0$$
	\end{cases}
	\end{displaymath}
	функция $\phi$ не обязана быть такой же, поэтому изменяется верхний индекс на $m$, эта функция также должна удовлетворять уравнению Гельмгольца  
	$$\Delta_\perp\phi^{m} + \kappa^2\phi^{m}=0$$
	Таким образом для системы уравнений Максвелла возможны два решения TM и TE волны. Но есть случай, когда поля чисто поперечны это случай TEM волны.
	Когда $\kappa = 0$, то есть $k=h$, продольные компоненты магнитного и электрического полей отсутствуют это и есть TEM волна.
	\begin{displaymath}
	e^{-ihz}\cdot
	\begin{cases}
	$$\displaystyle H_z = E_z = 0$$ \\
	$$\displaystyle\vec{E}_\perp = -\frac{1}{\epsilon\mu}\nabla_\perp\phi(\vec{r}_\perp)$$ \\
	$$\displaystyle\vec{H}_\perp = \frac{1}{\mu}[\nabla_\perp\phi(\vec{r}_\perp,\vec{z}_0)]$$ 
	\end{cases}
	\end{displaymath}
	$\phi$ -- поперечная волновая функция, удовлетворяющая уравнению $\Delta\phi=0$.
	Из формул выше видно, что поперечные компоненты полей удовлетворяют импедансному соотношению
	$$\vec{E}_\perp = \eta_{\perp\text{в}}[\vec{H}_\perp,\vec{z}_0]$$
	где $\eta_{\perp\text{в}}$ называется поперечным волновым сопротивлением
	$$\eta_{\perp\text{в}}=\sqrt{\frac{\mu}{\epsilon}} \left( \frac{k}{h} \right)^{\pm 1}$$ 
	где <<+>> -- соответствует волне типа TE, а <<-->> -- волне типа TM. Для TEM волны $\displaystyle \eta_{\perp\text{в}}=\sqrt{\frac{\mu}{\epsilon}}$
	
	\newpage
	\hypertarget{num4}{}
	В
\end{document}